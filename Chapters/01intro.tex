\chapter{Introduction}
\label{ch:intro}

This is a simple and hopefully easy to use (and alter according to different requirements) \LaTeX{} implementation according to the ACIT thesis handbook which can be found in ACIT Master class in Canvas. All of the text here is copied from the handbook except for \LaTeX{} specific notes. The handbook chapter guidelines may be commented out using \% so that you have the guidelines there while writing, but they are not visible in the compiled pdf-file. 
% Like this.

The chapter file naming is done so that they are alphabetical in order, and it is easy to remove or add more chapters before the results, discussion and conclusion.

\enquote{The introduction chapter will describe the background for your project work and clearly state the theme of the thesis, goals, hypotheses or research question(s), or the product being developed. The introduction may also include an overview for the reader, which describes the main structure of the report and clarifies any special factors.

The main goal of the background section is to clarify why the project’s research questions are relevant and which challenges they intend to address. The introduction should demonstrate the
student’s knowledge of their field of study and existing research, as well as of state of the art technologies} \citep{handbook}.




