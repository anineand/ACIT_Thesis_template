\chapter{Literature Review}
\label{ch:lit-review}

\enquote{Literature reviews present and critically analyse relevant works and the relationships between
them and place the thesis within a larger context by demonstrating how work undertaken in the
thesis seeks to fill a gap in, or extend, current knowledge.

Personal opinions and points of view should not be included in this chapter.

Correct and unambiguous references must accompany all information derived from academic
literature and other relevant sources.} (\cite{handbook})

\section{APA referencing}
The referencing style is set to APA as required, see \href{https://apastyle.apa.org/style-grammar-guidelines/references}{https://apastyle.apa.org/style-grammar-guidelines/references}. Note that APA has two citation styles:
\begin{itemize}
    \item \textbf{Parenthetical citation}, if the reference is not worked into the sentence, but in parenthesis at the end, use \texttt{\textbackslash citep\{key\}}: The results of this study were inconclusive \citep{winkler2018bdt}. 
    \item \textbf{Narrative or textual citation}, if the reference is worked into the text, use \texttt{\textbackslash citet\{key\}}: \citet{winkler2018bdt} determined that the results of the study were inconclusive.
    \item If the standard \texttt{\textbackslash cite\{key\}} is used, it results in a parenthetical citation without parenthesis: \cite{winkler2018bdt}.
\end{itemize}
To specify page (p.), pages (pp.) or chapter (Ch.), use \texttt{\textbackslash citep[here]\{citation-key\}} or \texttt{\textbackslash citet[here]\{citation-key\}}, e.g. \citet[pp. 2]{winkler2018bdt}.

The command \texttt{\textbackslash citetitle\{citation-key\}} can be used to cite the title, e.g. \citetitle{winkler2018bdt}.